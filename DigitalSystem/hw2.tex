\documentclass{article}

% Packages
\usepackage{amsmath} % For math equations
\usepackage{amssymb} % For math symbols
\usepackage{geometry} % For page layout
\usepackage{fancyhdr} % For headers and footers
\usepackage{lipsum} % For dummy text (remove this line)
\usepackage{float}
\usepackage{listings}
\usepackage{xcolor}
\usepackage{graphicx}
\usepackage{array} % 导入 array 宏包
\usepackage{verbatim} % 导入 verbatim 宏包
\usepackage{calc}
\usepackage{slashbox}
\usepackage{pict2e}

% Set the page margins
\geometry{margin=1in}
% Set the top margins
\addtolength{\topmargin}{-.5in}
\title{Digital System Assignment 2}
\author{He Tianyang}
\date{\today}


\begin{document}

\maketitle

\section*{Problem 1}
%1、用代数法化简逻辑函数
\textbf{Use algebraic method to simplify the logic function}
\begin{equation}
    Y = (\overline{A\overline{B}+\overline{A}B}\cdot C+A\overline{B}C)(AD+BC)
\end{equation}
\\
\textbf{Solves:}
\begin{align*}
    Y  = & (\overline{A\overline{B}+\overline{A}B}\cdot C+A\overline{B}C)(AD+BC) \\
    =    & ((AB+\overline{A}\cdot\overline{B})\cdot C+A\overline{B}C)(AD+BC)     \\
    =    & (ABC+\overline{A}\cdot\overline{B}C)(AD+BC)                           \\
    =    & ABCD+\overline{B}CD+ABC+\overline{A}C                                 \\
    =    & C(ABD+\overline{B}D+AB+\overline{A})                                  \\
    =    & C(AB+\overline{A}+\overline{B}D)                                      \\
    =    & \boxed{AC(B+D)}
\end{align*}

\section*{Problem 2}
%2. 化简逻辑函数Y(A,B,C,D) = \sigma m(3,4,5,7,8,9,10,11),约束条件: m0+m1+m2+m13+m14+m15=0
\textbf{Simplify the logic function}
\begin{equation}
    Y(A,B,C,D) = \sum m(3,4,5,7,8,9,10,11)
\end{equation}
\textbf{Constraint:}
\begin{equation}
    m_0+m_1+m_2+m_{13}+m_{14}+m_{15}=0
\end{equation}
\textbf{Solves:}
Solving this by drawing the Karnaugh map, we get:
\begin{table}[H]
    \centering
    \begin{tabular}{|c|c|c|c|c|}
        \hline
        \backslashbox{AB}{CD} & 00       & 01       & 11       & 10       \\ \hline
        00                    & $\times$ & $\times$ & 1        & $\times$ \\ \hline
        01                    & 1        & 1        & 1        &          \\ \hline
        11                    &          & $\times$ & $\times$ & $\times$ \\ \hline
        10                    & 1        & 1        & 1        & 1        \\ \hline
    \end{tabular}
\end{table}
So we get:
\begin{equation}
    Y = A\overline{B}+\overline{A}\ \overline{C}+\overline{A}CD
\end{equation}

\section*{Problem 3}
% 3、求解函数 Y
% Y = ( A ~C + A~B C + ~A BC ) ⊕ ( AB ~C + A~B C + ~ABC + ~A ~B ~C )

\textbf{Solve the function Y}
\begin{equation}
    Y = ( A \overline{C} + A\overline{B}C + \overline{A}BC ) \oplus ( AB\overline{C} + A\overline{B}C + \overline{A}\overline{B}\overline{C} )
\end{equation}
\textbf{Solves:}
drawing the Karnaugh map, we get:
\begin{table}[H]
    \centering
    \begin{tabular}{|c|c|c|c|c|}
        \hline
        \backslashbox{A}{BC} & 00 & 01 & 11 & 10 \\ \hline
        0                    & 1  &    &    &    \\ \hline
        1                    &    & 1  &    &    \\ \hline
    \end{tabular}
\end{table}
So we get:
\begin{equation*}
    \boxed{Y  = A\overline{B}C+\overline{A}BC}
\end{equation*}
\end{document}