\documentclass{article}

% Packages
\usepackage{amsmath} % For math equations
\usepackage{amssymb} % For math symbols
\usepackage{geometry} % For page layout
\usepackage{fancyhdr} % For headers and footers
\usepackage{lipsum} % For dummy text (remove this line)
\usepackage{float}


% Set the page margins
\geometry{margin=1in}
% Set the top margins
\addtolength{\topmargin}{-.5in}
\title{Digital System Assignment 1}
\author{He Tianyang}
\date{\today}


\begin{document}

\maketitle

\section*{Problem 1}
\textbf{Convert the complement and reverse code into signed decimal numbers.}\\
\\
\textbf{Solves:}

\begin{enumerate}
    \item $(01001001)_{reverse} = (00110110)_2 =2^5+2^4+2^2+2=\boxed{+54} $
    \item $(11100101)_{complement} = (11100101)_{reverse} = -(00011010)_2 = -(2^4+2^3+2)=\boxed{-26}$
\end{enumerate}

\section*{Problem 2}

\textbf{There are three temperature detectors. When the detected temperature exceeds 60°C, the output control signal is 1; if the detected temperature is below 60°C, the output control signal is 0. When two or more temperature detectors output 0, the main controller outputs a 1 signal to automatically control the regulating equipment to increase the temperature to above 60°C. Please write out the truth table and logical expression for the main controller.}\\
\\
\textbf{Solves:}\\
\\
Assume the three temperature detectors are $T_1$, $T_2$ and $T_3$, and the main controller is $Main$. The truth table is as follows:
\begin{table}[H]
    \centering
    \begin{tabular}{|c|c|c|c|}
        \hline
        $T_1$ & $T_2$ & $T_3$ & $Main$ \\ \hline
        0     & 0     & 0     & 1      \\ \hline
        0     & 0     & 1     & 1      \\ \hline
        0     & 1     & 0     & 1      \\ \hline
        0     & 1     & 1     & 0      \\ \hline
        1     & 0     & 0     & 1      \\ \hline
        1     & 0     & 1     & 0      \\ \hline
        1     & 1     & 0     & 0      \\ \hline
        1     & 1     & 1     & 0      \\ \hline
    \end{tabular}
\end{table}
Therefore, the logical expression for the main controller is:
\begin{align}
    Main & = \overline{T_1} \cdot \overline{T_2} \cdot \overline{T_3} + \overline{T_1} \cdot \overline{T_2} \cdot T_3 + \overline{T_1} \cdot T_2 \cdot \overline{T_3} + T_1 \cdot \overline{T_2} \cdot \overline{T_3} \\
         & = \sum m(0,1,2,4)
\end{align}
or
\begin{align}
    Main & = (T_1 + T_2 + T_3)\cdot (T_1 + T_2 + \overline{T_3})\cdot (T_1 + \overline{T_2} + T_3)\cdot (\overline{T_1} + T_2 + T_3) \\
         & = \prod M(3,5,6,7)
\end{align}

\end{document}