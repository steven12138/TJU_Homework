\documentclass{article}

% Packages
\usepackage{amsmath} % For math equations
\usepackage{amssymb} % For math symbols
\usepackage{geometry} % For page layout
\usepackage{fancyhdr} % For headers and footers
\usepackage{lipsum} % For dummy text (remove this line)
\usepackage{float}
\usepackage{listings}
\usepackage{xcolor}
\usepackage{graphicx}
\usepackage{array} % 导入 array 宏包
\usepackage{verbatim} % 导入 verbatim 宏包
\usepackage{calc}
\usepackage{tikz-qtree}
\usepackage{tikz}
\usepackage{forest}

\usetikzlibrary{positioning}

\definecolor{codegreen}{rgb}{0,0.6,0}
\definecolor{codegray}{rgb}{0.5,0.5,0.5}
\definecolor{codepurple}{rgb}{0.58,0,0.82}
\definecolor{backcolour}{rgb}{0.95,0.95,0.92}

\lstdefinestyle{mystyle}{
    commentstyle=\color{codegreen},
    keywordstyle=\color{magenta},
    numberstyle=\tiny\color{codegray},
    stringstyle=\color{codepurple},
    basicstyle=\ttfamily\footnotesize,
    breakatwhitespace=false,         
    breaklines=true,                 
    captionpos=b,                    
    keepspaces=true,                 
    numbers=left,                    
    numbersep=5pt,                  
    showspaces=false,                
    showstringspaces=false,
    showtabs=false,                  
    tabsize=2
}

\lstset{style=mystyle}
% Set the page margins
\geometry{margin=1in}
% Set the top margins
\addtolength{\topmargin}{-.5in}
\title{Algorithm Assignment 2}
\author{He Tianyang}
\date{\today}



\begin{document}
\maketitle

\section{Problem 1.}
% 归纳法(代入法)证明 Tn<=0,n=1|t[n/2]+t[n/2]+cn,n>1 <=cn[log_2n]
Use induction (substitution) to prove:
\begin{align}
    T(n) & = \begin{cases}
                 0,                                    & n = 1 \\
                 T(\frac{n}{2}) + T(\frac{n}{2}) + cn, & n > 1
             \end{cases} \\
         & \leq cn\log_2n
\end{align}

\textbf{Base case:} When $n = 1$,exist $c$ greater enough that satisfy $T(1) = 0 \leq c \cdot 1 \cdot \log_2 1$. The base case holds.

\textbf{Inductive step:} Assume that $T(k) \leq ck\log_2k$ for all $k < n$. We want to show that $T(n) \leq cn\log_2n$.

\begin{align*}
    T(n) & = T\left(\frac{n}{2}\right) + T\left(\frac{n}{2}\right) + c_2n              \\
         & \leq c_1\frac{n}{2}\log_2\frac{n}{2} + c_1\frac{n}{2}\log_2\frac{n}{2} + cn \\
         & = c_1n\log_2\frac{n}{2} + c_2n                                              \\
         & \leq c_1n\log_2n + c_2n                                                     \\
         & = O(n\log_2n)
\end{align*}

Therefore, by induction, $T(n) \leq cn\log_2n$ for all $n \geq 1$.

\section{Problem 2.}

Use the master theorem to solve the following recurrence relation:
\begin{equation}
    T(n) = 2T\left(\frac{n}{2}\right) + \Theta(n^{\frac{1}{2}})
\end{equation}

\textbf{Solution:}

The recurrence relation is of the form $T(n) = aT\left(\frac{n}{b}\right) + f(n)$, where $a = 2$, $b = 2$, and $f(n) = \Theta(n^{\frac{1}{2}})$. We can compare $f(n)$ with $n^{\log_ba}$ to determine which case of the master theorem applies.

In this case, $n^{\log_ba} = n^{\log_22} = n$. Since $f(n) = \Theta(n^{\frac{1}{2}})$, we have $f(n) = O(n^{1 - \epsilon})$ for $\epsilon = \frac{1}{2}$. Therefore, we are in case 1 of the master theorem, and the solution to the recurrence relation is:
\begin{equation*}
    \boxed{T(n) = n^{\log_ba} = n^{\log_22} = \Theta(n)}
\end{equation*}

\section{Problem 3.}

Expand the recurrence tree for the following recurrence relation,
and apply asymptotic analysis to determine the complexity of the recurrence relation.
\begin{equation}
    T(n) = T(2) + T(n - 2) + cn
\end{equation}

\textbf{Solution:}

The recurrence relation is of the form $T(n) = T(2) + T(n - 2) + cn$. We can expand the recurrence tree to visualize the recursive calls.
\begin{figure}[H]
    \centering
    \begin{forest}
        [T(n)
        [T(2)]
        [T(n-2)+cn
        [T(2)]
        [T(n-4)+c(n-2)
        [T(2)]
            [T(n-6)+c(n-4)
                [
                        T(2)
                    ]
                    [$\vdots$
                        [T(2)]
                            [T(2)+c4]
                    ]
            ]
        ]
        ]
        ]
    \end{forest}
    \label{fig:recurrence-tree}
    \caption{Recurrence tree for $T(n) = T(2) + T(n - 2) + cn$}
\end{figure}

Donated $h$ as the depth of the tree, the cost of each level is $cn$, where we have:
$$
    h=\frac{n}{2}
$$

The total cost of the tree is:
$$
    T(n)=hT(2)+\sum_{i=0}^{h-1}2c(i+1) = \frac{n}{2}\cdot T(2)+\frac{cn^2+4cn}{4} = \frac{n}{2}\cdot T(2)+\frac{c}{4}\cdot n^2+cn
$$

Compare the complexity of $\frac{n}{2}\cdot T(2)$ with $\frac{c}{4}\cdot n^2+cn$, we have:
\begin{equation*}
    T(n)  = \begin{cases}
        \Theta(nT(2)), & \text{if } T(2) = \Omega(n) \\
        \Theta(n^2),   & \text{if } T(2) = O(n)
    \end{cases}
\end{equation*}

\section{Problem 4.}
% 展开 T(n)=T(0.2n)+T(0.8n)+\Theta(n) 计算深度和渐进复杂度

Expand the recurrence tree for the following recurrence relation. Calculate the depth and the asymptotic complexity of the recurrence relation.
\begin{equation}
    T(n) = T(0.2n) + T(0.8n) + \Theta(n)
\end{equation}

\textbf{Solves:}

The recurrence relation is of the form $T(n) = T(0.2n) + T(0.8n) + \Theta(n)$. We can expand the recurrence tree to visualize the recursive calls.

Each $k$, $T(k) = T(0.2k) + T(0.8k) + \Theta(k)$, $T(0.8)$ is the larger part, and get the slowest asymptotic speed. Therefore, the depth of the recursive tree is $\log_{0.8}n$.

For the layer in depth $k$, the cost is $\Theta(n)$, and the total cost of the tree is:
\begin{equation*}
    T(n) = \sum_{i=0}^{\log_{0.8}n-1}2^i\Theta(n) = \Theta(n\log_{0.8}n)
\end{equation*}

Therefore, the asymptotic complexity of the recurrence relation is $\boxed{\Theta(n\log_{0.8}n)}$.

\section{Problem 5.}
Calculate the time complexity of algorithm A which is defined by the following recurrence relation:
\begin{equation}
    \begin{cases}
        f(n) = 2f(n-1)+1 \\
        f(0) = 1
    \end{cases}
\end{equation}

\textbf{Solves:}
Obviously, we have
\begin{align*}
    f(n) & = 2f(n-1)+1 \\
         & = 2(2f(n-2)+1)+1 \\
         & = 2^2f(n-2)+2+1 \\
         & = 2^3f(n-3)+2^2+2+1 \\
         & = \ldots \\
         & = 2^nf(0)+2^{n-1}+2^{n-2}+\cdots+2+1 \\
         & = 2^n+2^{n-1}+2^{n-2}+\cdots+2+1 \\
         & = 2^{n+1}-1
\end{align*}

Therefore, the time complexity of algorithm A is $\boxed{\Theta(2^n)}$.

\end{document}