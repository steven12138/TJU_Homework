\documentclass{article}
\usepackage{amsmath}
\addtolength{\topmargin}{-.5in}
\begin{document}

\title{Hw6, Algorithm}
\author{}
\date{}
\maketitle

\section{Q1. Knapsack Problem}

Given:
\begin{itemize}
    \item Profits \( P = [6, 3, 5, 4, 6] \)
    \item Weights \( w = [2, 2, 6, 5, 4] \)
    \item Capacity \( c = 10 \)
    \item Number of items \( n = 5 \)
\end{itemize}

\subsection*{Recursive Solution}
The recursive solution for the knapsack problem can be defined as:

\[
    K(n, c) =
    \begin{cases}
        0                                            & \text{if } n = 0 \text{ or } c = 0 \\
        K(n-1, c)                                    & \text{if } w[n-1] > c              \\
        \max\{K(n-1, c), P[n-1] + K(n-1, c-w[n-1])\} & \text{if } w[n-1] \leq c
    \end{cases}
\]

\subsubsection*{Base Case}
For \( n = 0 \) or \( c = 0 \):
\[
    K(0, c) = 0, \quad \forall c
\]

\subsubsection*{Recursive Case}
For \( n > 0 \) and \( c > 0 \):
\begin{align*}
    K(5, 10) & = \max\{K(4, 10), 6 + K(4, 6)\}                \\
    K(4, 10) & = \max\{K(3, 10), 4 + K(3, 5)\}                \\
    K(3, 10) & = K(2, 10) \quad (\text{since } w[2] = 6 > 10) \\
    K(2, 10) & = \max\{K(1, 10), 3 + K(1, 8)\}                \\
    K(1, 10) & = \max\{K(0, 10), 6 + K(0, 8)\} = 6            \\
    K(1, 8)  & = 6                                            \\
    K(2, 8)  & = \max\{K(1, 8), 3 + K(1, 6)\} = 9             \\
    K(3, 5)  & = \max\{K(2, 5), 5 + K(2, 0)\} = 5             \\
    K(2, 5)  & = \max\{K(1, 5), 3 + K(1, 3)\} = 6             \\
    K(4, 6)  & = \max\{K(3, 6), 4 + K(3, 1)\}                 \\
    K(3, 6)  & = \max\{K(2, 6), 5 + K(2, 0)\} = 9             \\
    K(2, 6)  & = \max\{K(1, 6), 3 + K(1, 4)\} = 9             \\
    K(4, 10) & = \max\{K(3, 10), 4 + 5\} = 10                 \\
    K(5, 10) & = \max\{10, 6 + 9\} = 15
\end{align*}

\subsection*{Dynamic Programming Solution}
\begin{align*}
    \text{Let } DP[i][j]     & = \text{Maximum profit using the first } i \text{ items and capacity } j \\
    DP[0][j]                 & = 0 \quad \forall j                                                      \\
    DP[i][0]                 & = 0 \quad \forall i                                                      \\
    \text{For } i = 1 \text{ to } n                                                                     \\
    \text{For } j = 1 \text{ to } c                                                                     \\
    \text{if } w[i-1] \leq j & \text{ then } DP[i][j] = \max(DP[i-1][j], P[i-1] + DP[i-1][j-w[i-1]])    \\
    \text{else } DP[i][j]    & = DP[i-1][j]
\end{align*}

\begin{table}[h!]
    \centering
    \begin{tabular}{c|ccccccccccc}
          & 0 & 1 & 2 & 3 & 4 & 5 & 6 & 7 & 8  & 9  & 10 \\
        \hline
        0 & 0 & 0 & 0 & 0 & 0 & 0 & 0 & 0 & 0  & 0  & 0  \\
        1 & 0 & 0 & 6 & 6 & 6 & 6 & 6 & 6 & 6  & 6  & 6  \\
        2 & 0 & 0 & 6 & 6 & 9 & 9 & 9 & 9 & 9  & 9  & 9  \\
        3 & 0 & 0 & 6 & 6 & 9 & 9 & 9 & 9 & 9  & 9  & 11 \\
        4 & 0 & 0 & 6 & 6 & 9 & 9 & 9 & 9 & 10 & 10 & 11 \\
        5 & 0 & 0 & 6 & 6 & 9 & 9 & 9 & 9 & 10 & 10 & 15 \\
    \end{tabular}
    \caption{DP Table for Knapsack Problem}
\end{table}

Thus, the maximum profit for capacity 10 is \( 15 \).

\section{Q2. 0/1 Knapsack Problem}

Let \( g(i, x) \) denote the maximum benefit for items \( 1, \ldots, i \) with a knapsack capacity of \( x \).

\subsection*{(1) Recurrence Relation}

The dynamic programming recurrence relation for the 0/1 knapsack problem is:

\[
    g(i, x) =
    \begin{cases}
        0                                        & \text{if } i = 0 \text{ or } x = 0 \\
        g(i-1, x)                                & \text{if } w_i > x                 \\
        \max\{g(i-1, x), p_i + g(i-1, x - w_i)\} & \text{if } w_i \leq x
    \end{cases}
\]

Where \( w_i \) is the weight and \( p_i \) is the profit of the \( i \)-th item.

\subsection*{(2) Example Calculation}

Given:
\begin{itemize}
    \item Number of items \( n = 4 \)
    \item Capacity \( c = 20 \)
    \item Weights \( w = [10, 15, 6, 9] \)
    \item Profits \( p = [2, 5, 8, 1] \)
\end{itemize}

We need to compute the DP table and backtrack to find the optimal solution.

\subsubsection*{DP Table}

\begin{table}[h!]
    \centering
    \begin{tabular}{c|cccccccccccccccccccccc}
          & 0 & 1 & 2 & 3 & 4 & 5 & 6 & 7 & 8 & 9 & 10 & 11 & 12 & 13 & 14 & 15 & 16 & 17 & 18 & 19 & 20 \\
        \hline
        0 & 0 & 0 & 0 & 0 & 0 & 0 & 0 & 0 & 0 & 0 & 0  & 0  & 0  & 0  & 0  & 0  & 0  & 0  & 0  & 0  & 0  \\
        1 & 0 & 0 & 0 & 0 & 0 & 0 & 0 & 0 & 0 & 2 & 2  & 2  & 2  & 2  & 2  & 2  & 2  & 2  & 2  & 2  & 2  \\
        2 & 0 & 0 & 0 & 0 & 0 & 0 & 0 & 0 & 0 & 2 & 2  & 2  & 2  & 2  & 5  & 5  & 5  & 5  & 5  & 7  & 7  \\
        3 & 0 & 0 & 0 & 0 & 0 & 0 & 8 & 8 & 8 & 8 & 8  & 8  & 8  & 8  & 8  & 10 & 10 & 10 & 10 & 13 & 13 \\
        4 & 0 & 0 & 0 & 0 & 0 & 0 & 8 & 8 & 8 & 8 & 8  & 8  & 8  & 8  & 8  & 10 & 10 & 10 & 10 & 13 & 13 \\
    \end{tabular}
    \caption{DP Table for the Given Example}
\end{table}

\subsubsection*{Backtracking to Find Optimal Items}

To find the items that make up the optimal solution, we backtrack from \( g(4, 20) \).

\begin{itemize}
    \item Start at \( g(4, 20) = 13 \)
    \item Since \( g(4, 20) = g(3, 20) \), item 4 is not included.
    \item Move to \( g(3, 20) = 13 \)
    \item Since \( g(3, 20) \neq g(2, 20) \), item 3 is included.
    \item Now, \( x = 20 - w_3 = 20 - 6 = 14 \)
    \item Move to \( g(2, 14) = 5 \)
    \item Since \( g(2, 14) = g(1, 14) \), item 2 is not included.
    \item Move to \( g(1, 14) = 2 \)
    \item Since \( g(1, 14) \neq g(0, 14) \), item 1 is included.
\end{itemize}

Thus, the optimal set of items to include are item 1 and item 3.


\section{Q3. Task Completion Problem}

Given \( n \) tasks numbered from \( 1 \) to \( n \) in topological order, each task \( i \) has two ways to complete:
\begin{itemize}
    \item Method 1: Cost \( C_{i,1} \) and Time \( T_{i,1} \)
    \item Method 2: Cost \( C_{i,2} \) and Time \( T_{i,2} \)
\end{itemize}

Define \( \text{cost}(i, j) \) as the minimum cost to complete tasks \( 1 \) to \( i \) within \( j \) time units.

The recurrence relation is:

\[
    \text{cost}(i, j) =
    \begin{cases}
        0                                                                                        & \text{if } i = 0                       \\
        \infty                                                                                   & \text{if } i > 0 \text{ and } j < 0    \\
        \min\{\text{cost}(i-1, j - T_{i,1}) + C_{i,1}, \text{cost}(i-1, j - T_{i,2}) + C_{i,2}\} & \text{if } i > 0 \text{ and } j \geq 0
    \end{cases}
\]

\subsection*{Base Cases}
\[
    \text{cost}(0, j) = 0 \quad \forall j \geq 0
\]
\[
    \text{cost}(i, j) = \infty \quad \forall i > 0 \text{ and } j < 0
\]

\subsection*{Recursive Case}
For \( i > 0 \text{ and } j \geq 0 \):
\[
    \text{cost}(i, j) = \min\{\text{cost}(i-1, j - T_{i,1}) + C_{i,1}, \text{cost}(i-1, j - T_{i,2}) + C_{i,2}\}
\]

This recurrence relation ensures that for each task \( i \), we consider both methods of completion and choose the one that minimizes the total cost while staying within the given time \( j \).

\section{Q4. Maximum Subset Sum Problem}

Given \( \sum_{i=1}^n s_i \geq c \), where \( \sum_{i \in J} s_i \leq c \) is an arbitrary positive integer, and \( J \) is a subset of \{1, 2, ..., n\}.

\subsection*{(1) Recurrence Relation}

Define \( \text{max\_sum}(i, j) \) as the maximum sum we can achieve using the first \( i \) elements without exceeding capacity \( j \).

The recurrence relation is:
\[
    \text{max\_sum}(i, j) =
    \begin{cases}
        0                                                                    & \text{if } i = 0 \text{ or } j = 0 \\
        \text{max\_sum}(i-1, j)                                              & \text{if } s_i > j                 \\
        \max\{\text{max\_sum}(i-1, j), s_i + \text{max\_sum}(i-1, j - s_i)\} & \text{if } s_i \leq j
    \end{cases}
\]

\subsection*{(2) Example to Illustrate the Algorithm}

Consider \( n = 4 \), \( c = 10 \), and \( s = [2, 3, 4, 5] \).

\subsubsection*{Step-by-Step Execution}

\begin{table}[h!]
    \centering
    \begin{tabular}{c|cccccccccccccccccccccc}
          & 0 & 1 & 2 & 3 & 4 & 5 & 6 & 7 & 8 & 9  & 10 \\
        \hline
        0 & 0 & 0 & 0 & 0 & 0 & 0 & 0 & 0 & 0 & 0  & 0  \\
        1 & 0 & 0 & 2 & 2 & 2 & 2 & 2 & 2 & 2 & 2  & 2  \\
        2 & 0 & 0 & 2 & 3 & 3 & 5 & 5 & 5 & 5 & 5  & 5  \\
        3 & 0 & 0 & 2 & 3 & 4 & 5 & 6 & 7 & 7 & 9  & 9  \\
        4 & 0 & 0 & 2 & 3 & 4 & 5 & 6 & 7 & 9 & 10 & 12 \\
    \end{tabular}
    \caption{DP Table for Maximum Subset Sum Example}
\end{table}

To find the subset that achieves the maximum sum not exceeding \( c \), backtrack from \( \text{max\_sum}(4, 10) \):

\begin{itemize}
    \item Start at \( \text{max\_sum}(4, 10) = 12 \)
    \item Since \( \text{max\_sum}(4, 10) \neq \text{max\_sum}(3, 10) \), item 4 is included.
    \item Now, \( j = 10 - s_4 = 10 - 5 = 5 \)
    \item Move to \( \text{max\_sum}(3, 5) = 5 \)
    \item Since \( \text{max\_sum}(3, 5) \neq \text{max\_sum}(2, 5) \), item 3 is included.
    \item Now, \( j = 5 - s_3 = 5 - 4 = 1 \)
    \item Move to \( \text{max\_sum}(2, 1) = 0 \), so no more items are included.
\end{itemize}

Thus, the optimal subset includes items 3 and 4, achieving the sum 9.


\begin{center}
    \begin{minipage}{0.48\textwidth}
        \vspace{-0.2cm}
        \centering
        \setlength\tabcolsep{0.043cm}%调列距
        \caption{\label{tab:cam}
            Classification error rate(\%) for ImageNet-to-ImageNet-C online CTTA task. Gain(\%) represents the percentage of improvement in model accuracy compared with the source method.}
        % \vspace{-0.3cm}
        \small
        % \setlength\tabcolsep{1pt}
        % \begin{adjustbox}{width=1\linewidth,center=\linewidth}
        \begin{figure*}[t]
            \centering
            \includegraphics[width=1\columnwidth]{Styles/figure/cam.pdf}
            \caption{}
            \label{fig:overview}
        \end{figure*}
        % \end{adjustbox}
        % \vspace{-0.2cm}
    \end{minipage}
    \hspace{0.02\textwidth}
    \begin{minipage}{0.48\textwidth}
        \vspace{-0.3cm}
        \centering
        \setlength\tabcolsep{0.1cm}%调列距
        \caption{\label{tab:tSNE}
            Average mIoU(\%) for the Citiscapes-to-ACDC. DI-A and DS-A represent the Domain-Invariant Adapter and the Domain-Specific Adapters. GNS means the Gradient Non-conflict Solver.}
        \small
        % \vspace{-0.3cm}
        % \setlength\tabcolsep{1pt}
        % \begin{adjustbox}{width=1\linewidth,center=\linewidth}
        \begin{figure*}[t]
            \centering
            \includegraphics[width=1\columnwidth]{Styles/figure/tsne.pdf}
            \caption{}
            \label{fig:overview}
        \end{figure*}
    \end{minipage}
\end{center}



\end{document}
