\documentclass{article}

\usepackage[utf8]{inputenc} % For Unicode characters
\usepackage{amsmath}        % For mathematical symbols
\usepackage{graphicx}       % For including images
\usepackage{hyperref}       % For hyperlinks
\usepackage{geometry}      % For setting page margins
\usepackage{float}
\usepackage{tcolorbox}

% Set the page margins
\geometry{margin=1in}
% Set the top margins
\addtolength{\topmargin}{-.5in}
\setlength{\parindent}{0pt} % 设置段落缩进为0

\title{Engineering Optics, Term Project}
\author{He Tianyang}
\date{\today}

\begin{document}

\maketitle

\section{Image-side focus point and focus length of the camera lens}

For the Camera lens 1, the structure parameters are as follows:
\begin{table}[H]
    \centering
    \begin{tabular}{cccc}
        \hline
        \textbf{Surface} & \textbf{Radius} & \textbf{Thickness} & \textbf{Glass}      \\
        \hline
        diaphragm        & 46.839          & 3.17               & n=1.51633, v=64.15  \\
        2                & 860.1441        & 0.1                &                     \\
        3                & 26.221          & 6.85               & n=1.51633, v=64.15  \\
        4                & 57.871          & 2.28               &                     \\
        5                & 124.006         & 14                 & n=1.74073, v=27.97  \\
        6                & 18.626          & 11.19              &                     \\
        7                & 43.776          & 2.42               & n=1.761441, v=26.55 \\
        8                & 121.402         & 51.27              &                     \\
        \hline
    \end{tabular}
    \caption{Structure parameters for Camera lens 1}
    \label{tab:camera1}
\end{table}

And we have the initial conditions:
\begin{align*}
    l_1 & = -\infty                                   \\
    h_1 & =1mm                                        \\
    i_1 & =\frac{h_1}{r_1}=\frac{1}{46.839}=0.0213497 \\
    n_0 & = 1
\end{align*}

Filling the table with the formulas
\begin{align}
    l_{k+1} & = l_k'-d_k                       \\
    i_k     & = \frac{(l_k-r_k)\cdot u_k}{r_k} \\
    i'_k    & = \frac{n_{k-1}\cdot i_k}{n_k}   \\
    u'_k    & = u_k+i_k-i'_k                   \\
    l'_k    & = r_k\cdot(1+\frac{i'_k}{u'_k})
\end{align}


\begin{table}[H]
    \centering
    \begin{tabular}{cccccc}
        \hline
        \textbf{Index} & $\mathbf{l}$ & $\mathbf{i}$ & $\mathbf{i'}$ & $\mathbf{u'}$ & $\mathbf{l'}$ \\
        \hline
        1              & $-\infty$    & 0.021350     & 0.014080      & 0.007270      & 137.554240    \\
        2              & 134.384240   & -0.006134    & -0.009301     & 0.010437      & 93.604411     \\
        3              & 93.504411    & 0.026782     & 0.017662      & 0.019557      & 49.902009     \\
        4              & 43.052009    & -0.005008    & -0.007594     & 0.022142      & 38.024545     \\
        5              & 35.744545    & -0.015760    & -0.009054     & 0.015436      & 51.273891     \\
        6              & 37.273891    & 0.015454     & 0.026901      & 0.003989      & 144.249128    \\
        7              & 133.059128   & 0.008135     & 0.004618      & 0.007505      & 70.713667     \\
        8              & 68.293667    & -0.003283    & -0.005783     & 0.010005      & 51.229261     \\
        \hline
    \end{tabular}
    \caption{Descriptive caption for the table}
\end{table}



Based on the above calculations, we have:
\begin{align*}
    l_8' & = 51.229261 \\
    u_8' & = 0.010005
\end{align*}

So we can know $\mathbf{l_F' = 51.229261mm}$, the image-side focus point is placed \textbf{51.229261mm} behind the last surface of the lens group.

And we can calculate the focus length $f$:
\begin{equation*}
    \boxed{
        f' = \frac{h}{u'} = \frac{1}{0.010005} =99.950025\mathbf{mm}
    }
\end{equation*}

Therefore, the vertex of the lens group is placed \textbf{99.950025mm} away from the image-side focus point. That is, the vertex is place \textbf{48.720764mm} away behind the last surface of the lens group.

\section{Object-side vertex and focus point}

Because of the reversibility of the light path, we can  reverse the lengs group. so the structure is as follows:

\begin{table}[H]
    \centering
    \begin{tabular}{cccc}
        \hline
        \textbf{Surface} & \textbf{Radius} & \textbf{Thickness} & \textbf{Glass}      \\
        \hline
        1                & 121.402         & 51.27              & n=1.761441, v=26.55 \\
        2                & 43.776          & 2.42               &                     \\
        3                & 18.626          & 11.19              & n=1.74073, v=27.97  \\
        4                & 124.006         & 14                 &                     \\
        5                & 57.871          & 2.28               & n=1.51633, v=64.15  \\
        6                & 26.221          & 6.85               &                     \\
        7                & 860.1441        & 0.1                & n=1.51633, v=64.15  \\
        diaphragm        & 46.839          & 3.17               &                     \\
        \hline
    \end{tabular}
    \caption{Reordered structure parameters for Camera lens 1}
    \label{tab:camera1_reversed}
\end{table}

We can use the same initial conditions:
\begin{align*}
    l_1 & = -\infty                                   \\
    h_1 & =1mm                                        \\
    i_1 & =\frac{h_1}{r_1}=\frac{1}{121.402}=0.008240 \\
    n_0 & = 1
\end{align*}

Filling the table with the same formulas, we have

\begin{table}[H]
    \centering
    \begin{tabular}{cccccc}
        \hline
        \textbf{Index} & $\mathbf{l}$ & $\mathbf{i}$ & $\mathbf{i'}$ & $\mathbf{u'}$ & $\mathbf{l'}$ \\
        \hline
        1              & $-\infty$    & 0.002135     & 0.001212      & 0.000923      & 280.839172    \\
        2              & 229.569172   & 0.003917     & 0.006900      & -0.002060     & -102.867777   \\
        3              & -105.287777  & 0.013702     & 0.007872      & 0.003771      & 57.505206     \\
        4              & 46.315206    & -0.002363    & -0.004113     & 0.005521      & 31.634481     \\
        5              & 17.634481    & -0.003839    & -0.002532     & 0.004214      & 23.104513     \\
        6              & 20.824513    & -0.000867    & -0.001315     & 0.004662      & 18.824167     \\
        7              & 11.974167    & -0.004597    & -0.003032     & 0.003096      & 18.027212     \\
        8              & 17.927212    & -0.001911    & -0.002898     & 0.004083      & 13.594514     \\
        \hline
    \end{tabular}
    \caption{Detailed data for the optical characteristics over different indices}
\end{table}

Based on the above calculations, we have:
\begin{align*}
    l_8' & = 13.594514 \\
    u_8' & = 0.004083
\end{align*}
That means, the object-side vertex is placed \textbf{13.594514mm} in front of the first surface of the lens group.

and we have the object-side focus length:
\begin{equation*}
    \boxed{
        f = -\frac{h}{u'} = \frac{1}{0.004083} =-244.999268\mathbf{mm}
    }
\end{equation*}

Therefore, the object-side focus point is placed \textbf{244.999268mm} in front of the object-side vertex. the vertex is placed \textbf{231.404754mm} in front of the first surface of the lens group.


\section{Imaging calculations}
the child is placed $d_1= -x_1-f= 9000mm, d_2=-x_2-f = 10000mm$ in front of the first surface of the lens group. The height of the child is $h_1 = 0.8\times 10^3mm$.

we can calculate the $x'_1$ and the $x'_2$ by the Newton Formula:

\begin{equation}
    x'_k = \frac{f\cdot f'}{x_k} = \frac{f\cdot f'}{f-d_k}
\end{equation}

Therefore, we have:
\begin{align*}
    x'_1 & = \frac{99.950025\cdot (-244.999268)}{99.950025-9000} = 2.75141\mathbf{mm} \\
    x'_2 & = \frac{99.950025\cdot(-244.999268)}{99.950025-10000} = 2.47349\mathbf{mm}
\end{align*}

So the image height of the child is:
\begin{align*}
    h'_1 & =  - \frac{x'}{f'}= -\frac{2.75141}{99.950025} = -0.02753\mathbf{mm} \\
    h'_2 & =  - \frac{x'}{f'}= -\frac{2.47349}{99.950025} = -0.02474\mathbf{mm}
\end{align*}
\end{document}