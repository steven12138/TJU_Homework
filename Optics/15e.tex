\documentclass{article}
\usepackage{amsmath}
\usepackage{algorithm}
\usepackage{algpseudocode}
\usepackage{tikz}

\begin{document}

\title{Optics HW 15-E}
\author{}
\date{}
\maketitle

\subsection*{1. Light coming from water}

To find the angle of incidence for complete polarization (Brewster's angle), we use Brewster's law:
\[
\tan \theta_w = \frac{n_{\text{flint}}}{n_{\text{water}}}
\]

Given:
\[
n_{\text{water}} = \frac{4}{3}, \quad n_{\text{flint}} = 1.72
\]

We need to solve for $\theta_w$:
\[
\theta_w = \arctan \left( \frac{n_{\text{flint}}}{n_{\text{water}}} \right)
\]

Substituting the values:
\[
\theta_w = \arctan \left( \frac{1.72}{\frac{4}{3}} \right) = \arctan \left( \frac{1.72 \times 3}{4} \right) = \arctan \left( \frac{5.16}{4} \right) = \arctan (1.29)
\]

Calculating the angle in degrees:
\[
\theta_w \approx 73.91^\circ
\]

\subsection*{2. Light coming from flint glass}

For light coming from the flint glass side:
\[
\tan \theta_f = \frac{n_{\text{water}}}{n_{\text{flint}}}
\]

We need to solve for $\theta_f$:
\[
\theta_f = \arctan \left( \frac{n_{\text{water}}}{n_{\text{flint}}} \right)
\]

Substituting the values:
\[
\theta_f = \arctan \left( \frac{\frac{4}{3}}{1.72} \right) = \arctan \left( \frac{4}{3 \times 1.72} \right) = \arctan \left( \frac{4}{5.16} \right) = \arctan (0.775)
\]

Calculating the angle in degrees:
\[
\theta_f \approx 44.42^\circ
\]

Therefore.
\begin{enumerate}
    \item The angle of incidence for complete polarization when light comes from the water side is approximately $73.91^\circ$.
    \item The angle of incidence for complete polarization when light comes from the flint glass side is approximately $44.42^\circ$.
\end{enumerate}

\end{document}
