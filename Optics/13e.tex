\documentclass{article}
\usepackage{amsmath}
\begin{document}

\title{Optics Homework 13-E}
\author{}
\date{}
\maketitle


\section{Q1}
To find the angular width of the central maximum, we use the single slit diffraction formula. The angular width of the central maximum is given by:
\[
\theta = \frac{\lambda}{a}
\]

Given:
\[
\lambda = 500 \times 10^{-9} \text{ m} \quad (\text{wavelength})
\]
\[
a = 1.50 \times 10^{-6} \text{ m} \quad (\text{slit width})
\]

Substituting the values:
\[
\theta = \frac{500 \times 10^{-9}}{1.50 \times 10^{-6}} = \frac{500}{1500} \times 10^{-3} = 0.333 \text{ radians}
\]

Since the angular width of the central maximum is from \(-\theta\) to \(\theta\), the total angular width is \(2\theta\):
\[
\text{Total angular width} = 2 \theta = 2 \times 0.333 = 0.666 \text{ radians}
\]

Converting this to degrees:
\[
\theta_{\text{deg}} = \theta \times \frac{180}{\pi} = 0.333 \times \frac{180}{\pi} \approx 19.1^\circ
\]

The angular width of the central maximum is approximately \(0.666\) radians or \(38.2^\circ\).

\section{Q2}
To find the position of the first minima, we use the single slit diffraction formula. The position of the first minima on the observing screen is given by:
\[
y = \frac{m \lambda L}{a}
\]
where:
\begin{itemize}
    \item \( m = 1 \) for the first minima
    \item \( \lambda = 580 \times 10^{-9} \text{ m} \) (wavelength)
    \item \( a = 0.30 \times 10^{-3} \text{ m} \) (slit width)
    \item \( L = 2.0 \text{ m} \) (distance to the screen)
\end{itemize}

Substituting the values:
\[
y = \frac{1 \times 580 \times 10^{-9} \times 2.0}{0.30 \times 10^{-3}} = \frac{580 \times 2.0}{0.30} \times 10^{-6} = \frac{1160}{0.30} \times 10^{-6} = 3866.67 \times 10^{-6} \text{ m}
\]

Converting this to millimeters:
\[
y = 3866.67 \times 10^{-6} \times 1000 = 3.87 \text{ mm}
\]

Therefore, the position of the first minima is approximately \(3.87 \text{ mm}\) from the center of the central maximum.


\section{Q3}
To determine which higher-order maxima are missing, we use the condition for missing maxima in a diffraction grating. The missing orders occur when the grating spacing \(d\) is an integer multiple of the slit width \(a\). This can be expressed as:
\[
m \lambda = a \sin \theta
\]
where:
\begin{itemize}
    \item \(m\) is the order of the maximum.
    \item \(a = 0.15 \text{ mm}\) (slit width).
    \item \(d = 0.6 \text{ mm}\) (slit spacing).
\end{itemize}

The missing orders occur when:
\[
\frac{d}{a} = \text{integer}
\]

Substituting the values:
\[
\frac{0.6 \text{ mm}}{0.15 \text{ mm}} = 4
\]

Therefore, the missing higher-order maxima are for the order:
\[
m = 4
\]

The higher-order maxima that are missing are the \(4\)th order maxima.

\section{Q4}
\subsection*{a) Separation between a set of slits}
To find the separation between the slits, we use the formula for the position of dark fringes in a diffraction pattern:
\[
y = \frac{L \lambda}{d}
\]
where:
\begin{itemize}
    \item \( \lambda = 450 \times 10^{-9} \text{ m} \) (wavelength)
    \item \( L = 1.80 \text{ m} \) (distance to the screen)
    \item \( y = 4.20 \times 10^{-3} \text{ m} / 2 = 2.10 \times 10^{-3} \text{ m} \) (distance from central maximum to one dark fringe)
\end{itemize}

Rearranging to solve for \( d \):
\[
d = \frac{L \lambda}{y}
\]
Substituting the values:
\[
d = \frac{1.80 \times 450 \times 10^{-9}}{2.10 \times 10^{-3}} = 3.857 \times 10^{-4} \text{ m} = 0.000386 \text{ m}
\]

\subsection*{b) Number of lines per meter}
The number of lines per meter \( N \) is the reciprocal of the slit separation \( d \):
\[
N = \frac{1}{d}
\]
Substituting the value of \( d \):
\[
N = \frac{1}{0.000386} \approx 2592.59 \text{ lines per meter}
\]


\section{Q5}
To find the size of the details that can be resolved by the telescope, we use the Rayleigh criterion for the minimum resolvable detail:
\[
\theta = 1.22 \frac{\lambda}{D}
\]
where:
\begin{itemize}
    \item \( \lambda = 500 \times 10^{-9} \text{ m} \) (wavelength)
    \item \( D = 7.6 \times 10^{-2} \text{ m} \) (aperture diameter)
    \item \( L = 12.5 \times 10^{3} \text{ m} \) (distance to the target)
\end{itemize}

Substituting the values:
\[
\theta = 1.22 \frac{500 \times 10^{-9}}{7.6 \times 10^{-2}} = 8.026 \times 10^{-6} \text{ radians}
\]

The linear size of the resolvable detail is given by:
\[
s = \theta \times L
\]
Substituting the values:
\[
s = 8.026 \times 10^{-6} \times 12.5 \times 10^{3} = 0.1003 \text{ m}
\]

Converting this to millimeters:
\[
s = 0.1003 \times 1000 = 100.3 \text{ mm}
\]


\end{document}

